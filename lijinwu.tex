%% start of file `template-zh.tex'.
%% Copyright 2006-2013 Xavier Danaux (xdanaux@gmail.com).
%
% This work may be distributed and/or modified under the
% conditions of the LaTeX Project Public License version 1.3c,
% available at http://www.latex-project.org/lppl/.


\documentclass[11pt,a4paper,sans]{moderncv}   % possible options include font size ('10pt', '11pt' and '12pt'), paper size ('a4paper', 'letterpaper', 'a5paper', 'legalpaper', 'executivepaper' and 'landscape') and font family ('sans' and 'roman')

% moderncv 主题
\moderncvstyle{classic}                        % 选项参数是 ‘casual’, ‘classic’, ‘oldstyle’ 和 ’banking’
\moderncvcolor{blue}                          % 选项参数是 ‘blue’ (默认)、‘orange’、‘green’、‘red’、‘purple’ 和 ‘grey’
%\nopagenumbers{}                             % 消除注释以取消自动页码生成功能

% 字符编码
\usepackage[utf8]{inputenc}                   % 替换你正在使用的编码
\usepackage{CJKutf8}

% 调整页面出血
\usepackage[scale=0.85]{geometry}
%\setlength{\hintscolumnwidth}{3cm}           % 如果你希望改变日期栏的宽度

% 个人信息
\name{}{李锦武}
%\title{简历题目 (可选项)}                     % 可选项、如不需要可删除本行
%\address{街道及门牌号}{邮编及城市}            % 可选项、如不需要可删除本行
%\phone[mobile]{爱尔兰: +353~87~168~7844}              % 可选项、如不需要可删除本行
\phone[mobile]{137~1033~6292(2013-09-14后启用)}              % 可选项、如不需要可删除本行
%\phone[fixed]{+2~(345)~678~901}               % 可选项、如不需要可删除本行
%\phone[fax]{+3~(456)~789~012}                 % 可选项、如不需要可删除本行
\email{ljw7630@hotmail.com}                    % 可选项、如不需要可删除本行
\homepage{about.me/lijinwu}                  % 可选项、如不需要可删除本行
%\extrainfo{附加信息 (可选项)}                 % 可选项、如不需要可删除本行
%\photo[64pt][0.4pt]{picture}                  % ‘64pt’是图片必须压缩至的高度、‘0.4pt‘是图片边框的宽度 (如不需要可调节至0pt)、’picture‘ 是图片文件的名字;可选项、如不需要可删除本行
%\quote{引言(可选项)}                          % 可选项、如不需要可删除本行

\renewcommand*\namefont{\fontsize{30}{30}\selectfont}

% 显示索引号;仅用于在简历中使用了引言
%\makeatletter
%\renewcommand*{\bibliographyitemlabel}{\@biblabel{\arabic{enumiv}}}
%\makeatother

% 分类索引
%\usepackage{multibib}
%\newcites{book,misc}{{Books},{Others}}
%----------------------------------------------------------------------------------
%            内容
%----------------------------------------------------------------------------------

\AtBeginDocument{
	\hypersetup{colorlinks,urlcolor=blue}
}

\begin{document}
\begin{CJK}{UTF8}{gbsn}                       % 详情参阅CJK文件包
\maketitle
\vspace*{-13mm}
\section{教育背景}
\cventry{2012-2013}{计算机科学, 理学硕士, 移动与普适计算方向}{圣三一学院}{爱尔兰都柏林}{}{}  % 第3到第6编码可留白

\cventry{2010-2012}{计算机科学理学荣誉学士(联合培养)}{都柏林理工学院}{都柏林}{\textit{\textbf{一等学位(first class honour})}}{}

\cventry{2008-2010}{软件工程(联合培养), 工学学士}{哈尔滨工业大学}{黑龙江哈尔滨}{\textit{\textbf{平均分92\%}}}{}

\section{研究经验(研究生毕业论文)}
\cvitem{题目}{\emph{Developing knowledge models of social media over World Wide Web} (使用网络爬虫、数据挖掘技术构建LinkedIn.com的知识图谱)}
\cvitem{导师}{\href{https://www.scss.tcd.ie/melike.sah/}{Dr. Melike Sah}, \href{https://www.scss.tcd.ie/vincent.wade/}{Prof. Vincent Wade}}
\cvitem{说明}{\small 使用Python编写爬虫抓取爱尔兰LinkedIn.com上面的个人公开简介和公司简介,利用数据挖掘方法进行除错和聚类分析,并使用语义网(\href{http://en.wikipedia.org/wiki/Semantic_Web}{Semantic Web})技术将之转换成RDF数据库. 该数据库被部署在亚马逊EC2云服务器上, 支持HTTP SPARQL查询}

\section{工作经验}
%\subsection{专业}
\cventry{2013.05-2013.09}{安卓和Rails开发工程师(全职实习)}{\href{http://popdeem.com/}{Popdeem}}{都柏林}{}{独立开发手机应用\href{https://play.google.com/store/apps/details?id=com.popdeem}{Popdeem}安卓版, 同时负责网站后台开发, 以及项目在AWS上的部署和配置, MYSQL数据库查询优化等
%\newline{}%
%工作内容:%
%\begin{itemize}%
%\item 工作内容 1;
%\item 工作内容 2、 含二级内容:
%  \begin{itemize}%
%  \item 二级内容 (a);
%  \item 二级内容 (b)、含三级内容 (不建议使用);
%    \begin{itemize}
%    \item 三级内容 i;
%    \item 三级内容 ii;
%    \item 三级内容 iii;
%    \end{itemize}
%  \item 二级内容 (c);
%  \end{itemize}
%\item 工作内容 3。
%\end{itemize}
}
\cventry{2012.11-2013.02}{ASP.NET开发工程师(兼职)}{\href{http://www.senddr.com/}{Senddr}}{都柏林}{}{网站\href{http://www.senddr.com/}{Senddr.com}的开发和维护, 如:支持OAuth登录, AJAX异步帐号验证, 以及网站的SEO%\newline{}说明行2
}
\cventry{2011.02-2011.08}{C\#软件工程师(全职实习)}{\href{http://sig.com/}{Susquehanna International Group(SIG/海纳集团)}}{都柏林}{}{高频证券交易数据的可视化, 为股市操盘手编写Wiki网站, 股票数据流日志的信息抓取和统计%\newline{}说明行2
}
%\subsection{其他}
%\cventry{年 -- 年}{职位}{公司}{城市}{}{说明}

\renewcommand*{\cvdoubleitem}[4]{%
 \cvline{#1}{\begin{minipage}[t]{\doubleitemmaincolumnwidth}#2\end{minipage}%
 \hfill%
 \begin{minipage}[t]{1.2\hintscolumnwidth}\raggedleft\hintfont{#3}\end{minipage}\hspace*{\separatorcolumnwidth}\begin{minipage}[t]{\doubleitemmaincolumnwidth}#4\end{minipage}}}
 
\renewcommand{\listitemsymbol}{-~}

\section{语言技能}
%\cvitemwithcomment{英语}{三年英语国家生活经验, 雅思7分}{}
%\cvitemwithcomment{普通话}{熟练}{}
%\cvitemwithcomment{粤语}{熟练}{}

\cvdoubleitem{英语}{三年英语国家生活经验, 雅思7分}{普通话\&粤语}{熟练}

\section{计算机技能}
\cvitem{编程语言}{C/C++, Java\&Android, C\#\&ASP.NET, HTML\&CSS\&Javascript, Python\&Django}
%\cvitem{数据库}{Oracle, SQL Server, MySQL}
%\cvitem{统计类工具}{Excel, Octave, R}
\cvdoubleitem{数据库}{Oracle, SQL Server, MySQL}{统计类工具}{Excel, Octave, R}
\cvitem{其他}{AWS EC2\&S3\&OpsWorks, Linux配置及命令行, 信息安全, 算法和问题求解}
%\cvdoubleitem{类别 3}{XXX, YYY, ZZZ}{类别 6}{XXX, YYY, ZZZ}

\section{获奖情况}
%\cvline{2011 \& 2012}{爱尔兰大学编程竞赛(ACM)第一名}
%\cvline{2009 \& 2010}{哈尔滨工业大学奖学金}
\cvdoubleitem{2011\&2012}{爱尔兰大学生编程竞赛(ACM)\textbf{第一名}}{2009\&2010}{哈尔滨工业大学奖学金}

%\section{个人兴趣}
%%\cvitem{足球}{\small 说明}
%\cvitem{人工智能}{\small 对机器学习, 自然语言识别和情景感知特别感兴趣, 有入门级别的知识并期待拥有继续学习的机会}
%\cvitem{编译原理}{\small 最近开始学习相关知识, 期望未来某一天通过自学可以实现自己发明的语言和IDE}

\section{项目经验}
\cvline{机器学习-\newline 收入预测}{使用了\href{http://archive.ics.uci.edu/ml/datasets/Census+Income}{UCI Census Income数据集}, 目标是根据人口普查信息预测一个人的月收入状况. 使用Python实现了朴素贝耶斯分类器和决策树分类器(并应用了顺序后向选择方法), 获得了84+\%的正确率 (官方网站最优结果为85.54\%)}
\cvline{数据可视化}{一个用于抓取服务器日志文件, 并显示统计图表的C\#桌面应用程序. 使用第三方UI库\href{http://www.devexpress.com/}{DevExpress}来简化开发流程及增强用户体验, 使用ADO.NET获取数据, 并且利用cache缓存来提高响应速度.}
\cvline{文件压缩}{一个基于哈夫曼树算法, 用C++实现的文件压缩工具(大二编程习作). \href{https://github.com/ljw7630/ForFun_FileCompressTool_HuffmanTree}{源代码}}

\renewcommand{\listitemsymbol}{-}             % 改变列表符号

%\section{其他 2}
%\cvlistdoubleitem{项目 1}{项目 4}
%\cvlistdoubleitem{项目 2}{项目 5\cite{book1}}
%\cvlistdoubleitem{项目 3}{}

% 来自BibTeX文件但不使用multibib包的出版物
%\renewcommand*{\bibliographyitemlabel}{\@biblabel{\arabic{enumiv}}}% BibTeX的数字标签
%\nocite{*}
%\bibliographystyle{plain}
%\bibliography{publications}                    % 'publications' 是BibTeX文件的文件名

% 来自BibTeX文件并使用multibib包的出版物
%\section{出版物}
%\nocitebook{book1,book2}
%\bibliographystylebook{plain}
%\bibliographybook{publications}               % 'publications' 是BibTeX文件的文件名
%\nocitemisc{misc1,misc2,misc3}
%\bibliographystylemisc{plain}
%\bibliographymisc{publications}               % 'publications' 是BibTeX文件的文件名

\clearpage\end{CJK}
\end{document}


%% 文件结尾 `template-zh.tex'.
